\documentclass[11pt, a4paper, twoside, article, openany]{memoir}

\usepackage[utf8]{inputenc}		% Dansk input encoding (tegn)
\usepackage[english, danish]{babel}		% Danske formuleringer / orddeling
\usepackage[T1]{fontenc}		% Output-indkodning af tegnsaet (T1)

%%% Memoir indstillinger
%% Afstand mellem afsnit og videre
%% NIX PILLE - medmindre strengt nødvendigt
\setaftersubsubsecskip{6pt}	
\setbeforesubsubsecskip{ 6pt}
%\setaftersubsecskip{6pt}
%\setbeforesubsecskip{-\baselineskip}
%\setaftersecskip{6pt}
%\setbeforesecskip{-\baselineskip}
%\setaftersecskip{1ex}

\raggedbottom


%\counterwithout{section}{chapter}
\chapterstyle{section}

% ¤¤ Marginer ¤¤ %
\setlrmarginsandblock{3.5cm}{2.5cm}{*}		% \setlrmarginsandblock{Indbinding}{Kant}{Ratio}
\setulmarginsandblock{3.0cm}{2.5cm}{*}		% \setulmarginsandblock{Top}{Bund}{Ratio}
\checkandfixthelayout

%%% Font valg %%%
\usepackage{mathpazo}	%Palatinofont - matematikformler
\usepackage{eulervm}		%Palatinofont

%%% FIGURER OG TABELLER %%%
\usepackage{graphicx} 						% Haandtering af eksterne billeder (JPG, PNG, PDF)

\usepackage[export]{adjustbox}

\usepackage{subfig}

\usepackage{multirow}                		% Fletning af raekker og kolonner (\multicolumn og \multirow)
\usepackage{colortbl} 						% Farver i tabeller (fx \columncolor, \rowcolor og \cellcolor)
\usepackage[dvipsnames]{xcolor}				% Definer farver med \definecolor. Se mere: http://en.wikibooks.org/wiki/LaTeX/Colors
%\usepackage{flafter}						% Soerger for at floats ikke optraeder i teksten foer deres erence
\usepackage{float}							% Muliggoer eksakt placering af floats, f.eks. \begin{figure}[H]
\usepackage{multicol}         	        	% Muliggoer tekst i spalter
%\usepackage{rotating}						% Rotation af tekst med \begin{sideways}...\end{sideways}
\usepackage{booktabs}
\usepackage{bigstrut}	% Excel2latex måskeh
\usepackage{tabularx}

%%% ¤¤ Matematik mm. %%%
\usepackage{amsmath,amssymb,stmaryrd} 		% Avancerede matematik-udvidelser
\usepackage{mathtools}						% Andre matematik- og tegnudvidelser
\usepackage{textcomp}                 		% Symbol-udvidelser (f.eks. promille-tegn med \textperthousand )
\usepackage{siunitx}						% Flot og konsistent praesentation af tal og enheder med \si{enhed} og \SI{tal}{enhed}
\sisetup{output-decimal-marker = {,}}		% Opsaetning af \SI (DE for komma som decimalseparator) 
\sisetup{exponent-product=\cdot, output-product=\cdot}	%Eksponent er gange tegn, output produkt er gange tegn
\sisetup{digitsep = none}					%Almindeligt komma - ingen mellemrum aka. til eurokomma

%%% MISC %%%
\usepackage{listings}						% Placer kildekode i dokumentet med \begin{lstlisting}...\end{lstlisting}
\definecolor{bg}{HTML}{F0F0F0}
\lstset{language=C++,
				showstringspaces = false,
				backgroundcolor = \color{bg},
                basicstyle=\ttfamily,
                keywordstyle=\color{blue}\ttfamily,
                stringstyle=\color{red}\ttfamily,
                commentstyle=\color{green}\ttfamily,
                morecomment=[l][\color{magenta}]{\#},
                extendedchars=true,
                numbers=left, numberstyle=\tiny,		% Linjenumre
                columns=flexible,						% Kolonnejustering
                breaklines, breakatwhitespace=true,		% Bryd lange linjer
                literate=%
                {æ}{{\ae}}1
                {å}{{\aa}}1
                {ø}{{\o}}1
                {Æ}{{\AE}}1
                {Å}{{\AA}}1
                {Ø}{{\O}}1
}

\usepackage{lipsum}							% Dummy text \lipsum[..]
\usepackage[shortlabels]{enumitem}			% Muliggoer enkelt konfiguration af lister
\usepackage{pdfpages}						% Goer det muligt at inkludere pdf-dokumenter med kommandoen \includepdf[pages={x-y}]{fil.pdf}	
\pdfoptionpdfminorversion=6					% Muliggoer inkludering af pdf dokumenter, af version 1.6 og hoejere

%	¤¤ Afsnitsformatering ¤¤ %
%\setlength{\parindent}{0mm}           		% Stoerrelse af indryk
\setlength{\parskip}{1.5mm}          		% Afstand mellem afsnit ved brug af double Enter
\linespread{1,1}							% Linie afstand

\usepackage{tikz}

% ¤¤ Visuelle  ¤¤ %
\usepackage[colorlinks]{hyperref}			% Danner klikbare referencer (hyperlinks) i dokumentet.
\hypersetup{colorlinks = true,				% Opsaetning af farvede hyperlinks (interne links, citeringer og URL)
	linkcolor = black,
	citecolor = black,
	urlcolor = black
}
\usepackage{url}

%%% REFERENCER %%%
%\usepackage{xr}
%\externaldocument{../dokumentation/dokumentation.tex}

%%% Referencer / Bibliografi %%%
\usepackage[backend=bibtex, sorting=none, style=numeric]{biblatex}
\bibliography{bibliography.bib}

\usepackage[draft, danish]{fixme}
\fxsetup{layout=footnote}

\graphicspath{{../fig/}{../fig}{fig/}{./}}

\usepackage{titlesec}

\setcounter{secnumdepth}{3}

\titleformat{\paragraph}
{\normalfont\normalsize\bfseries}{\theparagraph}{1em}{}
\titlespacing*{\paragraph}
{0pt}{3.25ex plus 1ex minus .2ex}{1.5ex plus .2ex}


%%%% Opsætning af dokument %%%%
\newcommand{\forfatter}{Gruppe xx}
\newcommand{\fag}{yy}
\newcommand{\titel}{zz}
\date{}

\author{\forfatter}
\title{\titel}


\setlength{\beforechapskip}{10pt}
\setlength{\afterchapskip}{10pt}
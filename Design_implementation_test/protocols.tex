\section{Protocols}

%TODO Write some more
This section describes the various protocols used throughout the system.

\subsection{Communication between server and client}

The server and client communicate using two .JSON files; toNav.JSON and fromNav.JSON. As the names suggest, one is used to send data from the client to the server (and from there to the navigation module), and one is used to send data from the server to the client. The contents of these two files is critical for both sides of the system to function properly, so having a standard format in each one is important. 

\subsubsection{toNav.JSON}

The file used to transfer data from the client to the server is used to send commands to the navigation module. This could be calculating a new coverage path for an area, telling the boat to stop, or setting the endpoint in the point to point page.
The protocol is based on JSON there for anything can be contained.

\begin{lstlisting}[caption = {Example of the least information in the toNav.JSON file}, captionpos=b, label={lst:toNavBasic}, language=json,firstnumber=1]
{
	"func_": "none"
}
\end{lstlisting}

The example show in listing \ref{lst:toNavBasic} shows the least amount of information that can be contained in the toNav.JSON file. the func_ name can hold a string, this is expected to be one of some predefined functions, in the example the string is \texttt{none}\\
The functions that can be inputted are; \texttt{none}, \texttt{calcP2P}, \texttt{calcCoverage}, \texttt{start}, and \texttt{stop}.
If one wants to send a \texttt{calcP2P} it is required that the write adds a container for a coordinate \texttt{target_position_} as in listing \ref{lst:toNavCalcP2P}. A coordinate is two number inputs one called \texttt{latitude_} and one called \texttt{longitude_}

\begin{lstlisting}[caption = {Example of a calcP2P call in the toNav.JSON}, captionpos=b, label={lst:toNavCalcP2P}, language=json,firstnumber=1]
{
	"func_": "calcP2P",
	"target_position_": {
		"latitude_":56.187317092640775,
		"longitude_":10.18372267484665
	}
}
\end{lstlisting}

\texttt{calcCoverage} needs two coordinates that describe a rectangle, and they are given in the format of a \texttt{coverage_rectangle_}, which contains a \texttt{start_coord_} and a \texttt{end_coord_} these are coordinates as in \texttt{calcP2P}. An example of this can be seen in listing \ref{lst:toNavCalcCoverage}.

\begin{lstlisting}[caption = {Example of a calcCoverage call in the toNav.JSON}, captionpos=b, label={lst:toNavCalcCoverage}, language=json,firstnumber=1]
{
	"func_": "calcCoverage",
	"coverage_rectangle_": {
		"start_coord_": {
			"latitude_": 56.17261819336624,
			"longitude_": 10.191444754600525
		},
		"end_coord": {
			"latitude_": 56.17248978086438,
			"longitude_": 10.191691517829895
		}
	}
}
\end{lstlisting}
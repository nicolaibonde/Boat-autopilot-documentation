%Her udpeges og diskuteres relevante dele af de opnåede resultater og deres betydning. Der skal også gives en samlet vurdering af de opnåede resultater med relation til projektets problemformulering. Der kan ligeledes være en opsummerende beskrivelse af resultater som I er særligt stolte af.
%Her udpeges og diskuteres relevante dele af de opnåede resultater og deres betydning. Der skal også gives en samlet vurdering af de opnåede resultater med relation til projektets problemformulering. Der kan ligeledes være en opsummerende beskrivelse af resultater som I er særligt stolte af.
\newpage
\chapter{Discussion}
This section is a discussion of the results from section 11. 

All functionality of the system has been verified in the acceptance test. However, as mentioned earlier, the project scope was so big that many hard choices had to be made underway to achieve this result.

A depth sensor was never attached to the final product, since it would require a lot of time to evalute another hardware component to purchase, order, and integrate in the system. It was decided early that this module would not be the focus, but that the system, and specifically the protocols used, should be so generic and robust that additions like this one could easily be made by a future team. The development team felt that there was great value in having a system that could easily be expanded upon or changed, and much less value in introducing additional hardware when time was limited because of the many mandatory components that had to be developed. 

In the same vein, polygon coverage was dismissed very early on due to its potentially immense complexity. Many projects taking considerably more time than ours and with much larger development teams are dedicated solely to developing efficient and robust polygon coverage algorithms. Developing an area coverage algorithm turned out to take considerable time by itself due to edge cases and making sure the algorithm wasn't too slow for the system to work, so rejecting polygon coverage from the outset seems to have been a very reasonable decision.

The question of whether to use files to hold the data exchanged between the controller, server, and website or finding another solution was debated in the group when coding the website early in development. It was decided that this simple solution would suffice since most of the system has no time-critical components (excluding the navigation algorithms themselves), since it relies upon user input, and a GPS which updates only once per second; a very long time relative to the speed of the processor. 

It was also possible to develop a function to save a path object to a file, which could then be read at a later time to repeat a coverage rectangle, very useful for a survey team studying a certain region of the sea floor over time. This functionality was, similar to others dicussed in this section, deemed unnecessary for the final product, but would have been added had there been more time.

On the navigation side of things, it was possible to have used great circle lines instead of rhumb lines. The decision to use rhumb lines was made strictly because is gives the user much better coherence between what she sees in the UI, and the path the system actually calculates. Rhumb lines become straight lines on a mercator projection, which is what the map uses to transform coordinates from an oblate spheroid and a rectangular map. Besides, it's rare that a survey is conducted across great distances, and so the advantages of using great circle lines are negligible.

The development team is overall pleased with the final product, and with having learned a great deal about tests, component choices, software design, and taking responsibility of the full development cycle as a small team. Making a wide variety of modules in several different languages all work together as intended was very satisfying, since it indicates that our decision to value generality and extendability over a full production-level implementation of the system (which we wouldn't have been able to finish in time) paid off.